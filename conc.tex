\chapter{Conclusions}
The chordate body plans is fairly conserved throughout the phyla, with most developing into a tadpole larvae containing a hollow dorsal neural tube, and a postanal tail containing a notochord flanked by muscle cells. 
\section{Comp}
Many non-model systems are now being sequenced with the drop in sequencing price. The methodology for assembly is not cut and dry, there are number step\textemdash quality trimming, filtering, choice of assembler and metrics to evaluate assembly. One of the standard is the N50, which has been designed for genomes and does not clearly translate to transcriptomes because of their fragmented nature, and the likelihood of chimeric contigs. We show that pre filtering quality trimmed assembly reads does not reduce the information content of the assembly. Additionally, the choice of assemblers returns similar results. Many times assembly methods are best decided on the usability of software and the availability of resources.

\section{Chap 4}



\section{Chapter 5}
Studying closely related organisms gives us insight into the underlying evolutionary mechanisms of divergence. 

Next generation sequencing is a great way to study non-model species. A broad swath of information can be gained from both RNA and DNA sequencing. This technology has allowed us to analyzed two closely related invertebrate chordates that also hybridize. This study has given us insight into the mechanisms behind tail-loss and development. Gene loss has occurred but other factors appear to be involved in tail-loss because expression from the tailed species appears to recover the loss features. 
