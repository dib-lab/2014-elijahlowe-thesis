\chapter{Literature Review}
\section{Ascidian tail development}

Ascidians are known for the bilateral and invariant cell cleavage. Their development well described up to the gastrulation stage\cite{nishida_cell_1983,nishida_cell_1985,nishida_cell_1987}. Like vertebrate chordates such as xenopus ascidians depend on maternally localized determinants to regulate cell moments and division, however the location and identity of these determinants are different although the development of the early body plans are similar \cite{lemaire_ascidians_2008}. In ascidians the first cell division is coordinated by  $\beta$-catenin which activates the vegetal gene and restrict GATA4/5 \cite{} and determines the axis of division \cite{}. 
The notochord is one of the most distinguishing characteristics of chordates. Solitary ascidians notochords typically come from two cell lineages, the primary notochord coming from the ``A'' blasomere and the secondary notochord comes from the ``B'' blastomere \cite{nishida_cell_1983}. At this stage the blastomeres are labeled in Conklin \cite{conklin_organization_1905} convention; ``a'' and ``A'' for the anterior animal and vegetal blastomeres, respectively and ``b'' and ``B'' for the posterior animal and vegetal blastomeres, respectively. Although the notochords cells have been traced back to this point, notochord inductions does not occur until the 32-cell stage, where the notochord/nerve chord precursors are activated by fiber growth factor (FGF) and without FGF activation the cells lose competency and the notochord can no longer from \cite{nakatani_basic_1996,nakatani_duration_1999}. 
By the 64-cell stage there are 10 notochord cell precursors, the 8 primary precursor notochord cells are identifiable and no longer multipotent, while the 2 secondary notochord cells are not restricted until the 110-cell stage \cite{nishida_cell_1985,yasuo_ascidian_1994,yasuo_conservation_1998,lemaire_unfolding_2009}. Two addition stages of cell division occur, one at gastrulation and one at neurulation, ending with 40 notochord cells, which is typical of most solitary ascidian tadpole larvae \cite{conklin_organization_1905}. At the onsite of neurulation the notochord begins to form, this process includes the closing of the neural tube and posterior movement of the notochord and muscle cells, followed by the polarization and intercalate mediolaterally to the midline through a process known as convergence and extension where the cells \cite{swalla_mechanisms_1993}. At this point the larval tail is constructed of a notochord flanked by 3 rows of muscles on each side, and both notochord and muscle cell derive from the same blastomeres \cite{nishida_cell_1985}. The arrangement of the notochord cells is a stochastic process, the anterior 32 cells\textemdash primary notochord cells\textemdash are always formed by the A7.3 and A7.7 blastomere and the posterior most 8\textemdash secondary\textemdash notochord cells are always formed by the B8.6 blastomere, but the ordering of the 32 most anterior is not determinate, cells from both the A7.3 and A7.7 intercalate in a random order \cite{nishida_cell_1983,nishida_cell_1985,miyamoto_formation_1985, swalla_mechanisms_1993,kourakis_one-dimensional_2014}. This process, along with muscle cell are the causes the larval tail to form \cite{miyamoto_formation_1985, jeffery_factors_1992,swalla_mechanisms_1993}.
Although a tailed larvae is typical of most ascidians, several species with in the Stolidobranchia order have individually undergone tail-loss, many of which fall in the Molgulidae \cite{berrill_studies_1931, jeffery_evolution_1999, huber_evolution_2000, maliska_molgula_2010}. The tail-less\textemdash anural\textemdash species develop in a similar manner and are indistinguishable from the tailed\textemdash urodele\textemdash counterparts up to late gastrulation \cite{berrill_studies_1931, swalla_interspecific_1990, jeffery_factors_1992}. Anural ascidians lack several urodele features including a converged and extended notochord, muscle cells and the otilith sensory organ. The absence of differentiated muscles cells and intercalated notochord are the cause for the lack of tail in these species \cite{miyamoto_formation_1985, swalla_interspecific_1990}. \textit{M. tectiformis} notochord cells do not divide again after the 10 precursor cells are formed and \textit{M. occulta} stops dividing after 20 cells \cite{jeffery_evolution_1999}. The same occurs in \textit{M. bleizi}, however after the 20 notochord cells are formed, the embryo attempts to make a tail but never does so \cite{swalla_novel_1993}. It has also been shown that chordate embryos without fully developed notochord and/or muscle cells do not fully elongate or fail completely to develop a tail \cite{jeffery_evolution_1999,takada_brachyury_2002,stemple_structure_2005}. 
Seeing that most ascidians have tailed larvae and that the tail can be restored through the use of interspecies hybrids, the lack of tail has been shown to be a loss of function. \textit{M. oculata} (urodel) and \textit{M. occulta} (anural) both of the Roscovita clade have been shown to produce hybrids in lab conditions. Of the known \textit{Molgula} species \textit{M. occulta} and \textit{M. oculata} are the only two that can hybridize. Although \textit{M. occulta} and \textit{M. oculata} have been found to dwell in the same habitat, hybrids have not been found in nature and have only been produced in lab conditions, and no other crosses are known to produce hybrids. Fertilizing \textit{M. oculata} eggs with \textit{M. occulta} sperm in most cases produce embryos with fully formed tails. The reciprocal hybrid produces an embryo with 20 notochord cells like \textit{M. occulta}, however the notochord cells converge and extent like \textit{M. oculata} \cite{swalla_interspecific_1990}. The ascidian tail has been shown to form in the presence of notochord and the absence of muscles cells \cite{miyamoto_formation_1985} and the hybrid tail is not flaked by muscles as that of tail species \cite{swalla_novel_1993}, however in hybrids embryos that express the p58 which is associated with cytoskeleton develop urodele features. Hybrid embryos that develop urodele features are batch specific and features are only restored in embryos that express p58 \cite{swalla_identification_1991,jeffery_factors_1992}. It was also shown that in hybrid embryos in which urodele features were restored, the number of cells that express acetylcholinesterase (AChE) in a vestigial muscle cell lineage increased \cite{jeffery_evolutionary_1991}. 

\section{Brachyury has been shown to be the  }

The induction of the notochord begins at the 32 cell stage by fibroblast growth factor (FGF) in the A6.2 and A6.4 notochord/nerve cord precursors\cite{satoh_ascidian_2001} after the 7th cleavage. FGF transducer FGF receptor, Ras, MEK and MAPK. MAPK promotes Ets which promotes \textit{Bra} at the 64 cell stage. It was observed from isolation experiments that notochord/nerve cord precursors that loss FGF competence at the 32 cell stage assume the default nerve cord cell fate \cite{minokawa_binary_2001}If FGF is not present at the 32 cell stage competence is loss and \textit{bra} is not induced. This is because is be MAPK which is down stream in the cascade is not activated and the induction of bra and repression of FoxB are not carried out \cite{hashimoto_transcription_2011}. And in the absence of bra notochord cell become nerve chord cells (Yasuo and Satoh 1998 Consevation of the developmental role of bra).As stated above the notochord is specific at the 64 cell stage. At this point \textit{brachyury} is expressed first weakly in the at the 64-cell stage in the notochord/nerve chord precursors \cite{yasuo_ascidian_1994} and unlike other chordates, in ascidians \textit{bra} is only expressed in the notochord cells \cite{yasuo_function_1993,corbo_characterization_1997,hotta_temporal_1999,takada_brachyury_2002}. Without \textit{bra} the ascidian tail does not form. Although \textit{bra} is necessary, its presence does not guarantee a tail. \textit{M. occulta} and \textit{M. tectiformis}, two tailless \textit{Molgula}, both express \textit{bra}. In both cases \textit{bra} expression stop earlier than that of \textit{M. oculata}, but produce different results. \textit{Bra} is expressed in the 10 precursor notochord cells in \textit{M. occulta}, another round of cell division occurs which does not in \textit{M. tectiformis}.  In these two species of \textit{Molgula} muscle actin became pseudo genes, however the mutation in the muscle actin genes are not the same \cite{swalla_novel_1993,jeffery_evolution_1999}. \textit{Manx} is another gene identified to be important for tell development in \textit{Molgula}, however, not in all ascidians. \textit{Manx} is lowly expressed in \textit{M. occulta}, and has been shown to restore the hybrid tail, but there is no homolog for \textit{manx} in \textit{C. intestinalis} \cite{swalla_requirement_1996,swalla_multigene_1999}. 
 
It was shown in \textit{H. roretzi} that \textit{FoxB} is represses the activation of \textit{bra} predominately through the binding of Fox BS1 (GCACTGA\textit{ACAAACA}TACATAG). \textit{FoxB} is activated by \textit{ZicN} and present in both nerve cord and notochords precursors, however is repressed by MAPK in the notochord cell lineage at the 64-cell stage \cite{hashimoto_transcription_2011}. MAPK is thought to be repressed by Ephin which is one of the key differences between notochord and nerve cord determination. Ephin and FoxB have redundant roles in the repression of the notochord fate, but differ in that ephin is spatial and FoxB is mediates temporal restriction of \textit{Bra} induction.
   
The Planar Cell Polarity (PCP) pathway is involved in cell movement during this process and mutations in \textit{prickle}\textemdash a known PCP gene\textemdash have shown to cause a shortened ascidian tail affecting both the mediolateral intercalation and the elongation of the ascidan tail\cite{jiang_ascidian_2005}. The \textit{pk} mutant \textit{aimless} produces a truncated tail, however the polarity of the nuclei are present, showing that prickle does not establish polarity with in the cell but polarity between cells, acting in a local manner and perhaps their is a global organizer \cite{jiang_ascidian_2005,kourakis_one-dimensional_2014}. However, even in the absence of the PCP pathway considerable convergence and elongation of the notochord was observed in Ciona, driven by a presumed boundary effect" \cite{veeman_chongmague_2008}.

Oikopleura did not exhibit the same mechanism for tail development as Ciona, of the 50 bra target genes previously identified cite only 26 of them had orthologs in Oikopleura \cite{kugler_evolutionary_2011} of those genes expression ranged from notochord specific to tail including possible notochord, to tissues that were clearly not the notochord.

------


there are 3 major pathways in chordates: FGF, BMP and Nodal.


\section{Assembling and analyzing data}
One of the major advances in science in the past 20 years was the implementation of sequencing technologies. These technologies allowed us to examine problems in ways not previously possible. The first wave were mircoarrays and sanger sequencing. Mircoarrays allow us to look at a wide spectrum of genes and understand relative expression within a sample. Sanger sequencing allowed us to sequence whole genomes and 
