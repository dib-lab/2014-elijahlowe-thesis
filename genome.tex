\chapter{Genome assembly and characterization}

\section{Materials and methods}

\subsection{Genomic DNA library preparation and sequencing}
Genomic DNA was phenol/chloroform extracted from dissected gonads of \textit{Molgula occulta} (Kupffer) and \textit{Molgula oculata} (Forbes) adults from Roscoff, France, and a \textit{Molgula occidentalis} (Traustedt) adult from Panacea, Florida, USA (Gulf Specimen Marine Lab). Genomic DNA was sheared using an M220 Focused-ultrasonicator (Covaris, Woburn, MA). Sequencing libraries were prepared using KAPA HiFi Library Preparation Kit (KAPA Biosystems, Wilmington, MA) indexed with DNA barcoded adapters (BioO, Austin, TX). Size selection was performed using Agencourt (Beckman?Coulter, Brea, CA) AMPure XP purification beads (300?400 bp fragments), or Sage Science (Beverly, MA) Pippin Prep (650?750 bp and 875?975 bp fragments). For \textit{M. occulta} and \textit{M. occidentalis} libraries, 6 PCR cycles were used. For \textit{M. oculata} libraries, 8 cycles were used for the 300?400 bp library, and 10 cycles were used for the 650?750 and 875?975 bp libraries. Libraries of different species but same insert size ranges were multiplexed for sequencing in three 100 � 100 PE lanes on a HiSeq 2000 sequencing system (Illumina, San Diego, CA) at the Genomics Sequencing Core Facility, Center for Genomics and Systems Biology at New York University (New York, NY). Thus, each lane was dedicated to a mix of species, specifically barcoded libraries of a given insert size range. Raw sequencing reads were deposited as a BioProject at NCBI under the ID\# PRJNA253689.
\subsection{Genome sequence assembly}
All genomes were assembled on Michigan State University High Performance Computing Cluster (http://contact.icer.msu.edu). Prior to assembly, read quality was examined using FastQC v0.10.1. Reads were then quality trimmed on both the 5? and 3? end using seqtk trimfq (https://github.com/lh3/seqtk) which uses Phred algorithm to determine the quality of a given base pair. Seqtk trimfq only trims bases, so no reads were discarded. Each library per species was then abundance filtered using 3-pass digital normalization to remove repetitive and erroneous reads \cite{brown_reference-free_2012,schwarz_genome_2013}, Howe et al., 2014). Genome assembly was done using velvet v1.2.08 (Zerbino and Birney, 2008) with k-mer overlap length (`k') ranging from 19 to 69 and scaffolding was done by Velvet, by default. Velvet does not produce separate files for contigs and scaffolds; because Velvet scaffolded conservatively, contigs dominated the assemblies so we refer to both contigs and scaffolds as contigs. CEGMA scores were then computed to evaluate genome completeness (Parra et al., 2007). The latest versions of three species' genome assemblies have been deposited on the ANISEED (Ascidian Network for In Situ Expression and Embryological Data) database for browsing and BLAST searching at http://www.aniseed.cnrs.fr/ (Tassy et al., 2010). Scripts for genome assembly and CEGMA analysis can be found in the following github repository: https://github.com/elijahlowe/molgula\textunderscore genome\textunderscore assemblies.git

\section{Results}
In addition to N50 lengths, we also used CEGMA (Core Eukaryotic Genes Mapping Approach) scores, in order to evaluate the assemblies' representative completeness (Parra et al., 2007). CEMGA reports scores for complete and partial alignments to a subset of core eukaryotic genes. An alignment is considered ?complete? if at least 70\% of a given protein model aligns to a contig in the assembly, while a partial alignment indicates that a statistically significant portion of the protein model aligns. The partial alignment scores are ?97\% or higher for all assemblies. \textit{M. oculata} has the best complete alignment score at ?90\%. \textit{M. occidentalis} and \textit{M. occulta} have complete alignment scores of 81\% and 77\% respectively (Table 1). These scores indicate that our assemblies contain at least partial sequences for the vast majority of protein-coding genes in the genomes of these species.

\subsection{Gene complexes}
When studying development it is important to characterize the genome for particular gene cluster/families. There are 13 HOX clusters, in humans  

