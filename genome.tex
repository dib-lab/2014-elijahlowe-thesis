\chapter{Genome assembly and characterization}
\section{Introduction}
Ascidians are now for there invariant cell lineage. Only very few solitary ascidians have deviated from there typical developmental something. Many of the genes have been studied across a number of ascidian, showing that gene function tends to be orthologous within the phyla. However there are cases where gene expression differences led to variation, in addition to genes with important functions . Genomics allows us to identify elements involved in the 
regulation of genes. To identify elements of such, information for a number of species 


\section{Materials and methods}

\subsection{Genomic DNA library preparation and sequencing}
Genomic DNA was phenol/chloroform extracted from dissected gonads of \textit{Molgula occulta} (Kupffer) and \textit{Molgula oculata} (Forbes) adults from Roscoff, France, and a \textit{Molgula occidentalis} (Traustedt) adult from Panacea, Florida, USA (Gulf Specimen Marine Lab). Genomic DNA was sheared using an M220 Focused-ultrasonicator (Covaris, Woburn, MA). Sequencing libraries were prepared using KAPA HiFi Library Preparation Kit (KAPA Biosystems, Wilmington, MA) indexed with DNA barcoded adapters (BioO, Austin, TX). Size selection was performed using Agencourt (Beckman?Coulter, Brea, CA) AMPure XP purification beads (300?400 bp fragments), or Sage Science (Beverly, MA) Pippin Prep (650-750 bp and 875-975 bp fragments). For \textit{M. occulta} and \textit{M. occidentalis} libraries, 6 PCR cycles were used. For \textit{M. oculata} libraries, 8 cycles were used for the 300?400 bp library, and 10 cycles were used for the 650-750 and 875-975 bp libraries. Libraries of different species but same insert size ranges were multiplexed for sequencing in three 100 � 100 PE lanes on a HiSeq 2000 sequencing system (Illumina, San Diego, CA) at the Genomics Sequencing Core Facility, Center for Genomics and Systems Biology at New York University (New York, NY). Thus, each lane was dedicated to a mix of species, specifically barcoded libraries of a given insert size range. Raw sequencing reads were deposited as a BioProject at NCBI under the ID\# PRJNA253689.
\subsection{Genome sequence assembly}
All genomes were assembled on Michigan State University High Performance Computing Cluster (http://contact.icer.msu.edu). Prior to assembly, read quality was examined using FastQC v0.10.1. Reads were then quality trimmed on both the 5' and 3' end using seqtk trimfq (https://github.com/lh3/seqtk) which uses Phred algorithm to determine the quality of a given base pair. Seqtk trimfq only trims bases, so no reads were discarded. Each library per species was then abundance filtered using 3-pass digital normalization to remove repetitive and erroneous reads \cite{brown_reference-free_2012,schwarz_genome_2013}, Howe et al., 2014). Genome assembly was done using velvet v1.2.08 (Zerbino and Birney, 2008) with k-mer overlap length (`k') ranging from 19 to 69 and scaffolding was done by Velvet, by default. Velvet does not produce separate files for contigs and scaffolds; because Velvet scaffolded conservatively, contigs dominated the assemblies so we refer to both contigs and scaffolds as contigs. CEGMA scores were then computed to evaluate genome completeness (Parra et al., 2007). The latest versions of three species' genome assemblies have been deposited on the ANISEED (Ascidian Network for In Situ Expression and Embryological Data) database for browsing and BLAST searching at http://www.aniseed.cnrs.fr/ (Tassy et al., 2010). Scripts for genome assembly and CEGMA analysis can be found in the following github repository: https://github.com/elijahlowe/molgula\textunderscore genome\textunderscore assemblies.git
\subsection{Gene identification and alignments}
Third-nine hox genes were identified and downloaded from the NCBI database. These sequences were then BLAST against each of the three assembled Molgula genomes. The alignments were then extracted BLAST against the NCBI non-redundant database. Alignments were annotated and placed in the following files, mocc_hox_aa.fa, mocu_hox_aa.fa, and moxi_hox_aa.fa, which are located ???. Hox12-13 were located on the same  

\section{Results}
Genomes of three Molgula species (M. occidentalis, M. oculata, and M. occulta) were sequenced using next-generation sequencing technology and assembled. A common metric for judging the quality of a genome assembly is the contig N50 length, which is determined such that 50\% of the assembly is contained in contigs of this length or greater. We used the contig N50 length to select the best assembly for each species given the varying ?k? parameter (length of k-mer overlap). A ?k? of 39 yields the best assembly for both M. occidentalis and M. occulta. The best ?k? for M. oculata was 61. M. occidentalis, M. occulta, and M. oculata N50 lengths were approximately 26.3 kb, 13 kb, and 34 kb, respectively (Table 1).
In addition to N50 lengths, we also used CEGMA (Core Eukaryotic Genes Mapping Approach) scores, in order to evaluate the assemblies' representative completeness (Parra et al., 2007). CEMGA reports scores for complete and partial alignments to a subset of core eukaryotic genes. An alignment is considered ?complete? if at least 70\% of a given protein model aligns to a contig in the assembly, while a partial alignment indicates that a statistically significant portion of the protein model aligns. The partial alignment scores are ?97\% or higher for all assemblies. \textit{M. oculata} has the best complete alignment score at ?90\%. \textit{M. occidentalis} and \textit{M. occulta} have complete alignment scores of 81\% and 77\% respectively (Table 1). These scores indicate that our assemblies contain at least partial sequences for the vast majority of protein-coding genes in the genomes of these species.
Various factors make it unreliable to predict genome size and gene density based on assembly metrics alone (Bradnam et al., 2013). Of the handful of sequences we isolated and analyzed, we found that the sizes of introns and upstream regulatory regions were roughly comparable to those from their Ciona orthologs. This suggests that the Molgula genomes may be as compact as the C. intestinalis genome (i.e., ?150?170 Mb, ?16,000 genes, Laird, 1971; Simmen et al., 1998; Satou et al., 2008).
Our sequencing efforts revealed extreme genetic divergence not only between Ciona and Molgula, as expected, but even within the Molgulids. For example, we used BLAST to identify the Molgula orthologs of C. intestinalis Mesp (Ciinte.Mesp, as per the proposed tunicate gene nomenclature rules, see Stolfi et al., 2014). Ciinte.Mesp is the sole ortholog of vertebrate genes coding for MesP and Mesogenin bHLH transcription factor family members (Satou et al., 2004). VISTA alignment shows high sequence similarity between sequences 5? upstream of the Mesp genes from the closely related M. oculata and M. occulta (Figure 1B). However, there is no conservation of Mesp DNA sequences, coding or non-coding, between M. oculata/occulta and M. occidentalis, nor between C. intestinalis and any of the three Molgula species (Figure 1?figure supplement 1). In previous phylogenetic surveys, M. occidentalis has been placed as an early-branching Molgula species, often grouped together in a subfamily with species ascribed to the genera Eugyra and Bostrichobranchus instead (Hadfield et al., 1995; Huber et al., 2000; Tsagkogeorga et al., 2009). Our sequencing results support the view that M. occidentalis is highly diverged from other Molgula spp.
\subsection{Gene complexes}
When studying development it is important to characterize the genome for particular gene cluster/families. There are 4 HOX clusters, in humans  totaling in 39 genes. Ciona has be found to have 9 box genes, Hox1-6, Hox10, and Hox12-13. Ciona is known to have to two clusters of hox genes across two chromosomes. Od also has 9 hox genes, hox1-2, hox4, a duplicate hox9, and hox10-13. Eight hox genes have be found in M. occulta and M. oculata, and nine have been found in M. occidentalis. Hox1-5, hox10 and hox12-13, with hox3-4 being found on the same contig in for both species. Additionally hox10, and hox12-13 are found on the same contig in M. oculata. However, it appears that the hox genes have been rearranged and hox10 is downstream of hox12-13. Hox12-13 are not found on the same contig in M. occulta, however when aligned with mVista there appears to be a strong case for synteny. The same set of hox genes were found in M. occidentalis, hox1-5, hox10 and hox12-13, however, there appears to be a duplicate hox10 gene \mytilde12kb apart found on the same contig. M. occidentalis hox genes span across 5 contigs, hox2 has a stop codon located in the 3-4 helix. 

