\chapter{Background}
Chordates are a branch of deuterostome that are characterized by a dorsal nervous system, pharyngeal gill slits, and defined by the presence of a notochord. Tunicates are one of the three subphyla of chordates and are grouped because of their outer covering known as a tunic. During development tunicates form a tailed larvae that closely resembles the vertebrate body plan \cite{jeffery_minireview_2002} and this tadpole larvae is typical of \textasciitilde 3000 tunicates \(cite\). Out of these 3000 species 16 are known to have independently lost their larval tail, with the majority of them being \textit{Molgula} \cite{berrill_studies_1931,swalla_interspecific_1990}. During this time, known as the free-swimming stage, the elongation and mobility of the tail is depended upon the proper formation of the notochord and muscle cells \cite{satoh_ascidian_2003}. As a tissue the notochord is closest related to cartilage and serves as the axial skeleton of the embryo in addition to a source patterning signaling \cite{jeffery_evolution_1999}. In ascidians and in lower vertebrae the improper formation of the notochord leads to severely shortened larva that cannot swim or feed properly \cite{di_gregorio_tail_2002,jiang_ascidian_2005,stemple_structure_2005}. We present a comparative study of the tailed \textit{M. oculata} and the tail-less \textit{M. occulta} through gene expression in order to understand the underlying factors behind tail development and tail loss.

Ascidians are a simpler system to study developmental processes, their development is well studied, they have invariant early cell lineages, a small number of cells \cite{lemaire_evolutionary_2011} and there has been no documentation of ascidians developing without an invariant cell lineage \cite{lemaire_ascidians_2008}. In \textit{Ciona intestinalis} there are ~2,600 cells, 36 of them being muscle, 40 of them being notochord and many of these cells have be traced starting at fertilization. Tunicates have a small number of cells compared to vertebrates, they also have rapid embryogenesis, compact genomes, few larval tissue types, simplified larval body plans and shallow gene networks \cite{corbo_characterization_1997,jeffery_minireview_2002,dehal_draft_2002}. For all of these reasons tunicates make great models for both tail development and loss, in addition to several Molgulids independently losing their tail and two of the Molgulids, a tailed and tailless species having the ability to hybridize \cite{jeffery_evolutionary_1991}. 

Tail development has been previously studied in ascidians and other chordates, with no one factor being the cause of a improperly form tail or lack of tail.