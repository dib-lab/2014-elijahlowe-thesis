\chapter{Introduction}
Chordates are a branch of deuterostomes that are characterized by a dorsal nervous system, pharyngeal gill slits, and defined by the presence of a notochord. Tunicates are one of the three subphyla of chordates and are so grouped because of their outer covering known as a tunic. During development tunicates form a tailed larvae that closely resembles the vertebrate body plan \cite{jeffery_minireview_2002}; this tadpole larvae is typical of \mytilde3000 tunicates \cite{huber_evolution_2000}. Out of these 3000 species, 16 are known to have independently lost their larval tail, with the majority of them being in the \textit{Molgula} clade \cite{berrill_studies_1931,swalla_interspecific_1990} and with only two tail-less species in the Styelidae \cite{huber_evolution_2000}. During the free-swimming larval stage, the elongation and mobility of the tail is depended upon the proper formation of the notochord and muscle cells \cite{satoh_ascidian_2003}. As a tissue the notochord is most closely related to cartilage and serves as the axial skeleton of the embryo in addition to a source of patterning signals \cite{jeffery_evolution_1999}. In ascidians and in lower vertebrates the improper formation of the notochord leads to severely shortened larva that cannot swim or feed properly \cite{di_gregorio_tail_2002,jiang_ascidian_2005,stemple_structure_2005}.
% @CTB: this next sentence seems out of context for the background?
We present a comparative study of the tailed \textit{M. oculata} and the tail-less \textit{M. occulta} through gene expression in order to understand the underlying factors behind tail development and tail loss.

Ascidians are a simple system in which to study developmental processes: their development is well studied, they have invariant early cell lineages and a small number of cells \cite{lemaire_evolutionary_2011}, and there has been no documentation of ascidians developing without an invariant cell lineage \cite{lemaire_ascidians_2008}. They also have rapid embryogenesis, compact genomes, few larval tissue types, simplified larval body plans and shallow gene networks \cite{corbo_characterization_1997,jeffery_minireview_2002,dehal_draft_2002}. Although this study present the first assembled \textit{Molgula} genomes, there are a number of sequenced tunicate genomes available: in particular, we use the assembled and annotated genome of \textit{Ciona intestinalis}, which serves as the most documented and closest complete reference for the \textit{Molgula} and other ascidian species \cite{dehal_draft_2002,satoh_ascidian_2003,satoh_ciona_2003}. In \textit{C. intestinalis} there are ~2,600 cells, with 36 muscle cells and 40 notochord.  Many of these cells have complete lineages traced starting at fertilization \cite{nishida_cell_1983}. Thus tunicates are good models for studying notochord specification.

In addition, several Molgulids have independently lost their tail, and two Molgulids, one tailed and one tail-less species, can be hybridized, offering the opportunity to study the genetics of tail loss \cite{jeffery_evolutionary_1991}. Although \textit{M. occulta} and \textit{M. oculata} present great systems evolutionarily to study tail development and loss, they have several shortcomings as experimental models: they are only found on the Northern coast of France and have yet to be cultured, they only spawn for one month out of the year, and many of the molecular techniques used in other ascidians have not yet been optimized for these two species.

Many genes in the notochord gene network have been identified by subtractive hybridization screening and microarrays \cite{jeffery_factors_1992,hotta_characterization_2000,gyoja_analysis_2007,kobayashi_differential_2013}. More recently, sequencing technologies such as Ion Torrent, Roche 454 and Illumina have made genome or transcriptome wide analysis more readily available for non-model species. These technologies have several advantages over microarrays: they have a wider scope, are more precise and are able to find novel genes \cite{marioni_rna-seq:_2008}. With the advances in technology we have now sequenced the transcriptomes of both species and their hybrid. This allows us to look at pivotal time points in tail development and compare across closely related species. This type of study has yet to be done. 

We began this project with RNA-seq data from several time points from each of the species (\textit{M. occulta} and \textit{M. oculata}) and their hybrid. Next-generation sequencing (NGS) is an effective method of producing observations and generating hypotheses to be tested experimentally. However, before we can make biological inferences from our data we have to produce a quality assembly. Because of this we first assessed the quality of an efficient low-memory assembly pipeline for our RNA-seq data and identified quality metrics other than N50 and contig length, since these are not best metrics for assessing transcriptome quality \cite{oneil_assessing_2013}. We later obtained genomic DNA and assembled the genomes of \textit{M. occulta}, \textit{M. oculata} and a more divergent species \textit{M. occidentalis}. This allowed us to analysis the homology between ascidian gene networks, and build more complete transcript modules for differential expression\cite{vijay_challenges_2012}.

DELETE TEXT BELOW UNLESS I CAN THINK OF A GOOD/USEFUL WAY TO INCORPORATE(but definitely put in first  sentence). 

Our study expands on prior knowledge, observing gene expression on a wider scale, the microarrays were done with isolate of the 64-cell which does not look at the whole picture seeing that the notochord develops through interactions with muscle and boundary effects \cite{keller_mechanisms_2000,veeman_chongmague_2008}.

Tail development has been previously studied in ascidians and other chordates, with no one factor being the cause of a improperly form tail or lack of tail. 

We started this project with RNA-seq data which presented us with the problem of determining which assembly was the best and what metrics should be used to analysis them.

Experimental techniques have yet to be adapted to \textit{M. occult} and \textit{M. oculata} because of there short gestation period, not being able to be cultured in lab conditions, although this is being currently developed amongst embryo specific difficulties. Most of the studies for tail development have been done in \textit{C. intestinalis} and \textit{H. roretzi}  
